\documentclass[xcolor=pdftex, dvipsnames, table]{beamer}

\RequirePackage[russian]{babel}
\RequirePackage[utf8]{inputenc}
\RequirePackage[TS1,T2A]{fontenc}
\usepackage{float}
\usepackage{amsmath}
\usepackage{listings}

\floatstyle{ruled}
\newfloat{program}{h}{src}
\floatname{program}{Листинг}

\usepackage{beamerthemeBoadilla}

\title{Система учёта и управления электронными налоговыми декларациями.}
\author{Залунин Павел}
\institute {
  БГУИР\\
  ФИТУ
}

\begin{document}

\frame{\titlepage}

\section{Введение}
\begin{frame}
  \frametitle{Система учёта и управления электронными налоговыми декларациями}
  Задачи, решаемые системой:
  \begin{itemize}
    \item \begin{Large}\textcolor{ForestGreen}{Автоматизация заполнения налоговых деклараций.}\end{Large}
    \item Предоставление возможности аудита заполненных деклараций.
    \item Предоставление возможности отправки деклараций в налоговые службы.
  \end{itemize}

  \vspace{3cm}
  \textbf{Существующие решения:} 1C Бухгалтерия, Парус, Галлактика ERP, Microsoft Dynamic NAV.
\end{frame}

\subsection{Налоговая декларация}
\begin{frame}
  \frametitle{Налоговая декларация}
  \begin{block}{Налоговая декларация}
    Документ, содержащий данные о финансовой деятельности предприятия.
  \end{block}
%  Существуют различные налоговые формы ( активные доходы, пассивные доходы, налог на $CO^{2}$ и др. )

  \begin{figure}
    \begin{center}
      \includegraphics[width=0.9\textwidth,maxheigth=0.9\textheight]{pics/declaration}
    \end{center}
  \end{figure}
\end{frame}

\section{Проектирование}
\subsection{Структура системы}
\begin{frame}
  \frametitle{Структура системы}
  \begin{figure}
    \begin{center}
      \includegraphics[width=0.7\textwidth,maxheight=0.8\textheight]{diagrams/system_struct}
    \end{center}
  \end{figure}
\end{frame}
%\subsection{Требования}
%\begin{frame}
%  \frametitle{Требования, предъявленные к системе}
%  \begin{itemize}
%    \item предоставление Web интерфейса;
%    \item редактирование декларации должно осуществляться интерактивно, без перезагрузки страницы;
%    \item использование реляционной БД;
%    \item возможность добавление новых деклараций и изменения существующих, необходимо предоставить простой способ для достижения этой цели;
%    \item на декларации может присутствовать три типа ячеек: a) формула b) интервальная c) пользовательская;
%    \item возможность задания произвольного значения для ячейки, которая считается по каким-либо правилам.
%  \end{itemize}
%\end{frame}
\begin{frame}
  \frametitle{Типы ячеек}
  На декларации могут присутствовать ячейки следующих типов:
  \begin{itemize}
    \item \textbf{Интервальная.} Значение для данной ячейки считается как сумма значений кредита либо дебета соответствующих <<счетов>> из <<Бухгалтерской Книги>>, находящихся в заданных интервалах.
    \item \textbf{Формула.} Значение для данной ячейки считается по формуле, в которой могут участвовать переменные, соответствующие другим ячейкам декларации. Примеры: \$\{A1\}+\$\{A3\}*ROUND(\$\{A4\}), SUM(A001:A999), \$\{D.C3\}-\$\{A3\}.
    \item \textbf{Пользовательская.} Значение для данной ячейки задает пользователь.
  \end{itemize}
\end{frame}

\section{Реализация}
\begin{frame}
  \frametitle{Реализация системы}
  \begin{enumerate}
    \item Выбор средств разработки ( язык Java )
    \item Реализация классов-сущностей.
    \item Реализация подсистемы управления (Spring MVC, Spring IoC)
    \item Реализация подсистемы управления декларацией.
    \item Создание схемы базы данных.
    \item Реализация подсистемы хранения. (Hibernate)
    \item Реализация подсистемы отображения декларации. (JSP, JavaScript, AJAX)
    \item Реализация подсистемы вычисления значения выражения.
    \item Тестирование системы. (JUnit)
  \end{enumerate}
\end{frame}
%\begin{frame}
%  \frametitle{Реализация подсистемы управления}
%  \begin{columns}
%    \begin{column}{0.6\textwidth}
%      \begin{center}
%        \begin{figure}
%          \includegraphics[scale=0.23]{diagrams/springmvc}
%          \caption{Обработка запроса Spring MVC}
%          \label{pic:springmvc}
%        \end{figure}
%      \end{center}
%    \end{column}
%    \begin{column}{0.5\textwidth}
%      Реализованные контроллеры:
%      \begin{itemize}
%        \item CompanyController
%        \item TaxKitController
%        \item TaxKitEditController
%        \item LedgerController
%        \item ReportStateController
%        \item ReportController
%      \end{itemize}
%      JSP Страницы:
%      \begin{itemize}
%        \item company/show.jsp
%        \item taxkit/list.jsp
%        \item report/report.jsp
%        \item ...
%      \end{itemize}
%    \end{column}
%  \end{columns}
%\end{frame}

\begin{frame}
  \frametitle{Реализация подсистемы хранения}
  \begin{block}{Объектно-реляционное отображение}
    Классы-сущности были отображены на соответствующие таблицы в базе данных, отображены связи между объектами. Типы данных Java отображены на типы данных SQL. Для реализации подсистемы хранения использовалась библиотека Hibernate.
  \end{block}

  \begin{figure}
    \includegraphics[width=0.6\textwidth]{../uml/ormsample}
    \caption{Отображение объектов и связей между ними на таблицы в базе данных}
  \end{figure}
\end{frame}

%\begin{frame}[fragile]
%  \frametitle{Реализация подсистемы вычисления выражений}
%  \begin{itemize}
%    \item Лексический Анализ (лексемы).
%    \item Синтаксический Анализ (АСД)
%    \item Интерпретация (значение выражения).
%  \end{itemize}

%  Подсистема вычисления выражений:
%  \begin{itemize}
%    \item Стандартные функции: abs, round, if, min, max
%    \item Возможность определения собственных функций
%    \item Поддерживаемые типы данных: string, double, long
%    \item Возможность подключения слушателей событий (загрузка переменных)
%  \end{itemize}

%\end{frame}

\begin{frame}
  \frametitle{Интерфейс пользователя. Редактирование декларации.}
  \begin{center}
    \includegraphics[width=0.9\textwidth]{pics/report_rules}
  \end{center}
\end{frame}


\section{Тестирование}
\subsection{Модульные тесты}
\begin{frame}
  \frametitle{Модульные тесты}
  В целях тестирования кода и контроля качества во время разработки были реализованы модульные тесты.
  Результат выполнения:
  \begin{center}
    \small {
      \rowcolors{1}{RoyalBlue!20}{RoyalBlue!5}
      \begin{tabular}{|m{5cm}|m{3cm}|m{2cm}|} \hline
        \textbf{Класс} &
        \textbf{Кол-во тестов} &
        \textbf{Результат} \\ \hline
        ReportFactoryImplTest & 1 & 100\% \\ \hline
        ReportManagerTest & 3 & 100\% \\ \hline
        ReportCrossInterpreterTest & 1 & 100\% \\ \hline
        ReportInterpreterTest & 1 & 100\% \\ \hline
        XMLReportLoaderTest & 1 & 100\% \\ \hline
        XMLCellLoaderTest & 2 & 100\% \\ \hline
        XMLFileReaderTest & 3 & 100\% \\ \hline
        AccountsInterpreterTest & 1 & 100\% \\ \hline
        ReflectionAssemblerTest & 1 & 100\% \\ \hline
        ExpressionParserTest & 2 & 100\% \\ \hline
        CompanyControllerTest & 2 & 100\% \\ \hline
        ReportStateControllerTest & 5 & 100\%  \\ \hline
        ReportsServiceImplTest & 1 & 100\%  \\ \hline
        TaxKitReportStateDAOImplTest & 1 & 100\%  \\ \hline
      \end{tabular}
    }
  \end{center}
\end{frame}

\section{Заключение}
\begin{frame}
  \frametitle{Заключение}
  Для выполнения дипломного проекта были решены следующие задачи:
  \begin{itemize}
    \item Анализ предметной области
    \item Проектирование системы
    \item Реализация системы
    \item Тестирование системы
    \item Выделение мер, обеспечивающих комфортные условия труда разработчикам систем документооборота
    \item Произведено технико-экономическое обоснование эффективности внедрения программного продукта
  \end{itemize}
\end{frame}

\begin{frame}
  \begin{center}
    \begin{huge}
      Спасибо за внимание.
    \end{huge}

    Вопросы, ответы.
  \end{center}
\end{frame}


\end{document}
