\documentclass[14pt,a4paper]{reportmod}
\usepackage[russian]{babel}
\usepackage{indentfirst}
\usepackage{amsmath}
\usepackage{graphicx}
\usepackage{rotating}
\usepackage{tabularx}
\usepackage{amsmath}
\usepackage{whiter4bbit}
\usepackage{multirow}
\usepackage{listings}
\usepackage{lastpage}
\usepackage{numberfigure}
\usepackage{longtable}
\usepackage{float}
\usepackage{verbatim}

\lstset{extendedchars=\true}
\frenchspacing

\renewcommand{\tabularxcolumn}[1]{m{#1}}

\begin{document}

\chapter*{Система учёта и управления электронными налоговыми декларациями.Руководство программиста}
\setcounter{chapter}{1}

\section{Структура директорий}
В качестве системы сборки используется Maven. Конфигурация сборки представлена в файле pom.xml каждого модуля.

Представим структуру директорий в которой содержится исходный код разработанной системы:
\verbatiminput{dir_struct}

Система состоит из трех модулей:
\begin{enumerate}
  \item autotax-engine - модуль, содержащий реализацию подсистемы управления декларацией, подсистемы отображения декларации;
  \item autotax-web - модуль, содержащий реализацию подсистемы управления, подсистемы хранения;
  \item expression-util - модуль, содержащий реализацию подсистемы вычисления выражений.
\end{enumerate}

\section{Описание Java классов}

\subsection{Загрузка декларации}
Для загрузки декларации из XML документа используется класс XMLReportLoader:
\begin{small}
\begin{verbatim}
/**
 * Конструктор
 *
 * @param registry реест деклараций
 * @param cellLoader загрузчик ячейки
 */
 public XMLReportLoader(XMLReportRegistry registry,
             XMLCellLoader cellLoader)

/**
 * реализация метода для загрузки структуры из
 * XML документа
 */
public Report loadReport(String name)
              throws ReportLoaderException
\end{verbatim}
\end{small}

XMLReportRegistry используется для получения местоположения (java.net.URL) XML документа, содержащего структуру декларации, XMLReportClassPathRegistry:

\begin{small}
\begin{verbatim}
/**
 * получить местоположение xml документа по имени
 */
public URL get(String reportName)
               throws RegistryException
\end{verbatim}
\end{small}

Класс XMLCellLoader позволяет создавать ячейки из представления в виде XML элементов:
\begin{small}
\begin{verbatim}
/**
 * конструктор
 *
 * @param registry реестр деклараций
 * @param typesRegistry реестр типов ячеек
 */
public XMLCellLoader(XMLReportRegistry registry,
                     CellsTypesRegistry typesRegistry)

/**
 * загружает все ячейки, которые содержатся в XML элементе
 *
 * @param cellsElement XML элемент, содержащий ячейки
 * @return множество ячеек
 * @throws CellLoaderException
 */
public Set<Cell> loadCells(Element cellsElement)
      throws CellLoaderException

/**
 * создание ячейки из XML элемента
 *
 * @param element XML элемент
 * @return ячейка
 * @throws CellLoaderException
 */
public Cell loadCell(Element element)
      throws CellLoaderException

/**
 * загрузка ячеек из XML документа
 */
public Set<Cell> loadCells(String reportName)
         throws CellLoaderException
\end{verbatim}
\end{small}

Класс CellsTypesRegistry инкапсулирует соответствие классов, представляющих тип ячейки, названием XML элементов, например (``type.formula'', ``type.editable'', ``type.accounts'')
\begin{small}
\begin{verbatim}
/**
 * ключ - название типа ячейки
 * значение - тип ячейки (класс-наследник CellType)
 *
 * @see CellType
 */
 private Map<String, Class<? extends CellType>> typesClasses;

/**
 * Конструктор
 * @param typesClasses карта, содержащая типы ячеек
 * @see #typesClasses
 */
public CellsTypesRegistry(Map<String,
              Class<? extends CellType>> typesClasses)

/**
 * Конструктор
 */
public CellsTypesRegistry()

/**
 * добавить новый тип
 *
 * @param type название типа
 * @param clazz тип
 */
public void registerType(String type,
              Class<? extends CellType> clazz)

/**
 * получить типы
 *
 * @return карта, ключ - название типа, значение - тип
 */
public Map<String,
           Class<? extends CellType>> getTypesClasses()

/**
 * возвращает тип ячейки по имени
 */
public Class<? extends CellType>
              get(String type) throws RegistryException
\end{verbatim}
\end{small}

Пример использования классов для загрузки декларации (исходный код модульного теста):
\begin{small}
\begin{verbatim}
public void testLoadReport() throws Exception {
   XMLReportRegistry registry =
          new XMLClassPathReportRegistry();

   CellsTypesRegistry cellsTypesRegistry =
          new CellsTypesRegistry();
   cellsTypesRegistry.
          registerType("formula",
                      CellTypeFormula.class);
   cellsTypesRegistry.
          registerType("accounts",
                      CellTypeAccounts.class);
   cellsTypesRegistry.
         registerType("editable",
                      CellTypeEditable.class);
   cellsTypesRegistry.
         registerType("constant",
                     CellTypeConstant.class);

   ReportLoader loader =
         new XMLReportLoader(registry,
             new XMLCellLoader(registry,
                           cellsTypesRegistry));
   Report report = loader.loadReport("states");

   assertNotNull(report);
   assertNotNull(report.getCells());
   assertEquals(report.getCells().size(), 5);
}
\end{verbatim}
\end{small}

\subsection{Управление декларацией}
Для управления декларацией с момента её создания используется класс ReportManager, опишем методы и поля данного класса:

\begin{small}
\begin{verbatim}
/**
 * декларация
 */
private Report report;

/**
 * интерпретатор декларации
 */
private ReportInterpreter reportInterpreter;

/**
 * преобразователь строковых значений в соответствующее
 * @see CellValue
 */
private ValueConverter valueConverter =
    new ValueConverterImpl();

/**
 * конструктор
 *
 * @param report декларация
 */
public ReportManager(Report report)

/**
 * конструктор
 * @param report декларация
 * @param contextFactory фабрика для создания контекста
 */
public ReportManager(Report report,
    CrossReportContextFactory contextFactory)

/**
 * подсчитать декларацию
 *
 * @return подсчитанная декларация
 */
public Report calculateReport()

/**
 * функция возвращает ячейку по имени
 */
private F<String, Cell> cellForName

/**
 * пересчитать ячейку с новым состоянием
 *
 * @param cellStateDTO состояние
 * @return пересчитанные ячейки (зависимые)
 */
public List<Cell>
       recalculateCell(CellStateDTO cellStateDTO)

/**
 * пересчитать ячейку
 *
 * @param name название ячейки
 * @return пересчитанные ячейки (зависимые)
 */
public List<Cell> recalculateCell(String name)

/**
 * пересчитать ячейку с новым значением
 *
 * @param name название ячейки
 * @param cellValue новое значение
 * @return пересчитанные ячейки (зависимости)
 */
public List<Cell> recalculateCell(String name,
                             CellValue cellValue)

/**
 * изменить тип ячейки
 *
 * @param cellState состояние
 */
public void changeCellType(CellStateDTO cellState)

/**
 * изменить тип ячейки
 * @param cell ячейка
 * @param activeType название нового типа
 * @return ячейка
 */
public Cell changeCellType(Cell cell,
                String activeType)

/**
 * изменить тип ячейки
 * @param cellName название ячейки
 * @param activeType название нового типа
 * @return ячейка
 */
public Cell changeCellType(String cellName,
                 String activeType)

/**
 * получить значение из строкового представления
 * @see #valueConverter
 *
 * @param cellStateDTO состояние ячейки
 * @return значение
 */
public CellValue createValue(CellStateDTO cellStateDTO)

/**
 * получить декларацию
 *
 * @return декларация
 */
public Report getReport()

\end{verbatim}
\end{small}

Опишем интерфейс CellInterpeter, используемый для подсчета ячеек:
\begin{small}
\begin{verbatim}
/**
 * CellVisitor
 * Date: 10.03.2010
 *
 * Интерпретатор ячейки
 */
public interface CellInterpreter<A extends CellType> {
/**
 * посчитать значение ячейки
 *
 * @param cellType тип ячейки
 * @param context контекст
 * @return значение ячейки
 */
LazyValue<CellValue> calculate(A cellType, Context context);
}
\end{verbatim}
\end{small}

Метод ``calculate'' возвращает экземпляр класса LazyValue, который содержит реализацию интерфейса F0 для подсчета соответсвующей ячейки. Класс LazyValue используется для реализации отложенных вычислений. Значение (в данном случае - экземпляр класса CellValue) высчитывается по требованию, в процессе подсчета декларации - после того, как получены экземпляры классов LazyValue для всех ячеек. Для получения значение, требуется вызвать метод load.
\begin{small}
\begin{verbatim}
public abstract class LazyValue<A> {
/**
 * функция, которая реализует возвращает
 * значение, представленное экземпляром класса
 * A
 *
 * @return функция
 */
public abstract F0<A> value();

/**
 * результат выполнения функции
private A b = null;

 /**
  * возвращает значение, которое
  * посчитано функцией
  *
  * @return значение
  */
public A load(){
   if(b==null){
        b = value().f();
    }
    return b;
}
}
\end{verbatim}
\end{small}

Опишем интерфейс F0:
\begin{small}
\begin{verbatim}
public interface F0<A> {
/**
 * @return значение
 */
 A f();
}
\end{verbatim}
\end{small}

\subsection{Классы подсистемы хранения}
Подсистема хранения была реализована с использованием библиотеки Hibernate, которая предоставляет средства объектно-ориентированного отображения. Для хранения объекта в базе данных требуется определить аннотацию ``@Entity''для класса-сущности. Для определения первичного ключа требуется определить аннотацию ``@Id'' для поля, которое является первичным ключом.


Пример декларации аннотаций в классе-сущности Company:
\begin{small}
\begin{verbatim}
@Entity
@Proxy(lazy=false)
public class Company {
@Id
@GeneratedValue(strategy=GenerationType.SEQUENCE)
private Long id;

private String name;

private String code;

public Long getId() {
   return id;
}

public void setId(Long id) {
   this.id = id;
}

public String getName() {
   return name;
}

public void setName(String name) {
   this.name = name;
}

public String getCode() {
   return code;
}

public void setCode(String code) {
   this.code = code;
}
}

\end{verbatim}
\end{small}

Подсистема хранения предоставляет объекты доступа к данным, которые предоставляют методы, необходимые для загрузки, сохранения, обновления таблиц, соответствующих классам-сущностям. Приведем абстрактных класс, от которого наследуются все классы, предоставляющие методы доступа к данным:
\begin{small}
\begin{verbatim}
public abstract class GenericHibernateDAO<Entity> {

/**
 * фабрика для сессии бибилиотеки
 * Hibernate
 */
private SessionFactory sessionFactory;

public SessionFactory getSessionFactory() {
   return sessionFactory;
}

public void setSessionFactory(SessionFactory sessionFactory) {
   this.sessionFactory = sessionFactory;
}

public Session session(){
    return sessionFactory.getCurrentSession();
  }
}
\end{verbatim}
\end{small}

Представим интерфейс объекта доступа к данным ( Entity - класс, от которого наследуются все классы-сущности).
\begin{small}
\begin{verbatim}
public interface DAO<Entity> {

/**
 * загрузка объекта
 * @param id идентефикатор объекта
 * @return экземпляр класса Entity
 **/
Entity load(Long id);

/**
 * сохранения объекта
 * в базе данных
 * @param entity экземпляр объекта
 **/
void insert(Entity entity);

/**
 * обновление объекта
 * в базе данных
 * @param entity экземпляр класса Entity
 **/
void update(Entity entity);

/**
 * удаление объекта из базы данных
 * @param id идентификатор объекта
 **/
void remove(Long id);

/**
 * получить список всех объектов
 * хранящихся в базе данных
 **/
List<Entity> list();
}
\end{verbatim}
\end{small}

Реализация интерфейса для работы с классом ReportState:
\begin{small}
\begin{verbatim}
public class TaxKitReportStateDAOImpl
    extends GenericHibernateDAO<TaxKitReportState>
   implements TaxKitReportStateDAO{
/**
 * найти состояние декларации, по названию и уникальному
 * идентификатору налоговых форм
 * @param name название декларации
 * @param taxKitId идентификатор налоговых форм
 **/
@Override
@SuppressWarnings("unchecked")
public TaxKitReportState findByNameAndTaxKit(String name, Long taxKitId) {
 List<TaxKitReportState> states =
  session().createCriteria(TaxKitReportState.class).
      add(Restrictions.eq("taxKit.id", taxKitId)).
       add(Restrictions.eq("name", name)).list();
  TaxKitReportState reportState = null;
  if(states.size()>0){
     reportState = states.get(0);
  }
  return reportState;
}

/**
 * добавить запись в таблицу базы данных
 * соответствующую объекту TaxKitReportState
 * @param tkReportState состояние декларации
 **/
@Override
public void insert(TaxKitReportState tkReportState) {
   session().save(tkReportState);
}

/**
 * обновить запись в таблице
 * @param tkReportState состояние декларации
 **/
@Override
public void store(TaxKitReportState tkReportState) {
   session().update(tkReportState);
}
}
\end{verbatim}
\end{small}

\subsection{Классы подсистемы управления}
Подсистема управления содержит классы, представляющие собой контроллеры Spring MVC. Контроллеры содержат методы, обрабатывающие запросы.

Приведем исходный код контроллера, обрабатывающего запросы на отображение декларации:
\begin{small}
\begin{verbatim}
@Controller
@RequestMapping("/taxkits/{taxKitId}/reports")
@SessionAttributes({"reportManager", "taxKit"})
public class ReportController {

@Autowired
private Assembler<Report, ReportDTO> reportAssembler;

@Autowired
private ReportsService reportsService;

@Autowired
private TaxKitService taxKitService;

private void storeSelectedTaxKit(Model model){
  TaxKit taxKit = taxKitService.getSelected();
  if(taxKit!=null){
     model.addAttribute("taxKit", taxKit.getId());
     model.addAttribute("selected", true);
   } else {
     model.addAttribute("selected", false);
   }
}

/**
 * метод обрабатывающий запрос
 * на получение списка доступных деклараций
 * осуществляется переход на jsp страницу, которая определена
 * под именем "report.index"
 * @param model экземпляр модели, используемый для
 * передачи параметров на jsp страницу
 **/
@RequestMapping("/")
public String report(Model model){
   storeSelectedTaxKit(model);
   model.addAttribute("reports", reportsService.availableReports());
   return "report.index";
}

/**
 * метод обрабатывающий запрос на отображение
 * декларации
 * @param reportName название декларации, которую требуется
 * отобразить
 * @param taxKitId идентификатор налоговых форм, для которых
 * отображается декларация
 * @param model экземпляр модели, который используется для
 * передачи параметров на jsp страницу
 **/
@RequestMapping("/{name}/show")
public String show(@PathVariable("name") String reportName,
                   @PathVariable("taxKitId") Long taxKitId,
                   Model model){
  storeSelectedTaxKit(model);

  Report report = reportsService.loadReport(reportName, taxKitId);
  model.addAttribute("report", reportAssembler.assemble(report));
  model.addAttribute("reports", reportsService.availableReports());
  return "report.show";
}
}
\end{verbatim}
\end{small}

В рамках данной подсистемы реализованы следующие контроллеры:
\begin{itemize}
  \item CompanyController
  \item LedgerController
  \item ReportController
  \item ReportStateController
  \item TaxKitController
  \item TaxKitCreateController
  \item TaxKitEditController
\end{itemize}


\section{Описание XML документа}

Приведем описание XML документа, который содержит структуру декларации. Опишем правило составления XML документа на языке XML Schema:

\verbatiminput{xml/rep.xsd}

\end{document}
